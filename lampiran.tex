%-----------------------------------------------------------------------------%
\addappendix{CHANGELOG}
\chapter*{Lampiran 1: CHANGELOG}
\label{appendix:changelog}
%-----------------------------------------------------------------------------%
\todo{Silakan hapus lampiran ini ketika Anda mulai menggunakan \f{template}.}

\f{Template} versi terbaru bisa didapatkan di \url{https://gitlab.com/ichlaffterlalu/latex-skripsi-ui-2017}. Daftar perubahan pada \f{template} hingga versi ini:
\begin{itemize}
	\item versi 1.0.3 (3 Desember 2010):
		\begin{itemize}
			\item \f{Template} Skripsi/Tesis sesuai ketentuan \f{formatting} tahun 2008.
			\item Bisa diakses di \url{https://github.com/edom/uistyle}.
		\end{itemize}
	\item versi 2.0.0 (29 Januari 2020):
		\begin{itemize}
			\item \f{Template} Skripsi/Tesis sesuai ketentuan \f{formatting} tahun 2017.
			\item Menggunakan BibTeX untuk sitasi, dengan format \f{default} sitasi IEEE.
			\item \f{Template} kini bisa ditambahkan kode sumber dengan \f{code highlighting} untuk bahasa pemrograman populer seperti Java atau Python.
		\end{itemize}
	\item versi 2.0.1 (8 Mei 2020):
		\begin{itemize}
			\item Menambahkan dan menyesuaikan tutorial dari versi 1.0.3, beserta cara kontribusi ke template.
		\end{itemize}
	\item versi 2.0.2 (14 September 2020):
		\begin{itemize}
			\item Versi ini merupakan hasil \f{feedback} dari peserta skripsi di lab \f{Reliable Software Engineering} (RSE) Fasilkom UI, semester genap 2019/2020.
			\item BibTeX kini menggunakan format sitasi APA secara \f{default}.
			\item Penambahan tutorial untuk \code{longtable}, agar tabel bisa lebih dari 1 halaman dan header muncul di setiap halaman.
			\item Menambahkan tutorial terkait penggunaan BibTeX dan konfigurasi \f{header}/\f{footer} untuk pencetakan bolak-balik.
			\item Label "Universitas Indonesia" kini berhasil muncul di halaman pertama tiap bab dan di bagian abstrak - daftar kode program.
			\item \f{Hyphenation} kini menggunakan \code{babel} Bahasa Indonesia. Aktivasi dilakukan di \code{hype-indonesia.tex}.
			\item Minor adjustment untuk konsistensi \f{license} dari template.
		\end{itemize}
	\item versi 2.0.3 (15 September 2020):
		\begin{itemize}
			\item Menambahkan kemampuan orientasi \f{landscape} beserta tutorialnya.
			\item \code{\bslash{}captionsource} telah diperbaiki agar bisa dipakai untuk \code{longtable}.
			\item Daftar lampiran kini telah tersedia, lampiran sudah tidak masuk daftar isi lagi.
			\item Nomor halaman pada lampiran dilanjutkan dari halaman terakhir konten (daftar referensi).
			\item Kini sudah bisa menambahkan daftar isi baru untuk jenis objek tertentu (custom), seperti: "Daftar Aturan Transformasi". Sudah termasuk mekanisme \f{captioning} dan tutorialnya.
			\item Perbaikan minor pada tutorial.
		\end{itemize}
	\item versi 2.1.0 (8 September 2021):
		\begin{itemize}
			\item Versi ini merupakan hasil \f{feedback} dari peserta skripsi dan tesis di lab \f{Reliable Software Engineering} (RSE) Fasilkom UI, semester genap 2020/2021.
			\item Minor edit: "Lembar Pengesahan", dsb. di daftar isi menjadi all caps.
			\item Experimental multi-language support (Chinese, Japanese, Korean).
			\item Support untuk justifikasi dan word-wrapping pada tabel.
			\item Penggunaan suffix "(sambungan)" untuk tabel lintas halaman. Tambahan support suffix untuk \code{\bslash{}captionsource}.
		\end{itemize}
\end{itemize}

%-----------------------------------------------------------------------------%
\addappendix{Judul Lampiran 2}
\chapter*{Lampiran 2: Judul Lampiran 2}
\label{appendix:sample}
%-----------------------------------------------------------------------------%
Lampiran hadir untuk menampung hal-hal yang dapat menunjang pemahaman terkait tugas akhir, namun akan mengganggu \f{flow} bacaan sekiranya dimasukkan ke dalam bacaan.
Lampiran bisa saja berisi data-data tambahan, analisis tambahan, penjelasan istilah, tahapan-tahapan antara yang bukan menjadi fokus utama, atau pranala menuju halaman luar yang penting.

%-----------------------------------------------------------------------------%
\section*{Subbab dari Lampiran 2}
\label{appendix:sampleSubchap}
%-----------------------------------------------------------------------------%
\todo{Isi subbab ini sesuai keperluan Anda. Anda bisa membuat lebih dari satu judul lampiran, dan tentunya lebih dari satu subbab.}
