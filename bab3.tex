%-----------------------------------------------------------------------------%
\chapter{\babTiga}
%-----------------------------------------------------------------------------%
Bab ini menjelaskan tentang hal-hal \f{advanced} dalam \latex. Hal ini mencakup bagaimana cara menulis persamaan matematis di \latex, menambahkan daftar isi, catatan, PDF, menambahkan kode, bahkan menambahkan perintah baru.

\todo{Sejatinya bab ini digunakan untuk membahas inti dari penelitian Anda. Sesuaikan saja dengan kebutuhkan Anda: misalkan bab empat Anda adalah penjelasan terkait desain sistem.}


%-----------------------------------------------------------------------------%
\section{Membuat Persamaan Matematis}
%-----------------------------------------------------------------------------%
Di \latex, kita dapat membuat persamaan matematis baik yang terdiri dari satu persamaan maupun lebih dari satu persamaan. Anda bisa mencoba mengikuti dan memahami contoh kode yang ada di \f{template} ini untuk kebutuhan tugas akhir Anda. Menggunakan \latex juga perlu latihan dan lihai memahami dokumentasi.

%-----------------------------------------------------------------------------%
\subsection{Satu Persamaan}
%-----------------------------------------------------------------------------%

\noindent \begin{align}\label{eq:garis}
	\cfrac{y - y_{1}}{y_{2} - y_{1}} = 
	\cfrac{x - x_{1}}{x_{2} - x_{1}}
\end{align}

\equ~\ref{eq:garis} diatas adalah persamaan garis. 
\equ~\ref{eq:garis} dan \ref{eq:bola} sama-sama dibuat dengan perintah \code{\bslash{}align}. 
Perintah ini juga dapat digunakan untuk menulis lebih dari satu persamaan. 

\noindent \begin{align}\label{eq:bola}
	\underbrace{|\overline{ab}|}_{\text{pada bola $|\overline{ab}| = r$}} 
	= \sqrt[2]{(x_{b} - x_{a})^{2} + (y_{b} - y_{a})^{2} + 
		\vert\vert(z_{b} - z_{a})^{2}}
\end{align}

%-----------------------------------------------------------------------------%
\subsection{Lebih dari Satu Persamaan}
\label{sec:multiEqu}
%-----------------------------------------------------------------------------%
\noindent \begin{align}\label{eq:matriks}	
	|\overline{a} * \overline{b}| &= |\overline{a}| |\overline{b}| \sin\theta 
	\\[0.2cm]
	\overline{a} * \overline{b} &=  
	\begin{array}{| c c c |}
		\hat{i} & x_{1} & x_{2} \\
		\hat{j} & y_{1} & y_{2} \\
		\hat{k} & z_{1} & z_{2} \\
	\end{array} \nonumber \\[0.2cm]
	&= \hat{i} \,
	\begin{array}{ | c c | }
		y_{1} & y_{2} \\
		z_{1} & z_{2} \\
	\end{array} 
	+ \hat{j} \,
	\begin{array}{ | c c | }
		z_{1} & z_{2} \\
		x_{1} & x_{2} \\
	\end{array} 
	+ \hat{k} \,	
	\begin{array}{ | c c | }
		x_{1} & x_{2} \\
		y_{1} & y_{2} \\
	\end{array}
	\nonumber
\end{align}

Pada \equ~\ref{eq:matriks} dapat dilihat beberapa baris menjadi satu bagian 
dari \equ~\ref{eq:matriks}. 
Sedangkan dibawah ini dapat dilihat bahwa dengan cara yang sama, \equ~
\ref{eq:gabungan1}, \ref{eq:gabungan2}, dan \ref{eq:gabungan3} memiliki nomor 
persamaannya masing-masing. 

\noindent \begin{align}\label{eq:gabungan1}	
	\int_{a}^{b} f(x)\, dx + \int_{b}^{c} f(x) \, dx = \int_{a}^{c} f(x) \, dx
	\\\label{eq:gabungan2}
	\lim_{x \to \infty} \frac{f(x)}{g(x)} = 0 \hspace{1cm} 
	\text{jika pangkat $f(x)$ $<$ pangkat $g(x)$} \\\label{eq:gabungan3}
	a^{m^{a \, ^{n}\log b }} = b^{\frac{m}{n}}
\end{align}



%-----------------------------------------------------------------------------%
\section{Mengubah Tampilan Teks}
%-----------------------------------------------------------------------------%
Beberapa perintah yang dapat digunakan untuk mengubah tampilan adalah: 
\begin{itemize}
	\item \bslash{}f \\
	Merupakan alias untuk perintah \bslash textit, contoh 
	\f{contoh hasil tulisan}.
	\item \bslash{}bi \\
	\bi{Contoh hasil tulisan}.
	\item \bslash{}bo \\
	\bo{Contoh hasil tulisan}.
	\item \bslash{}m \\
	\m{Contoh hasil tulisan}.
	\item \bslash{}mc \\
	\mc{Contoh hasil tulisan}.
	\item \bslash{}code \\ 
	\code{Contoh hasil tulisan}.
\end{itemize}


%-----------------------------------------------------------------------------%
\section{Menambahkan Kode Program}
%-----------------------------------------------------------------------------%
Pada \latex, kode program seringkali disebut \f{listing}. Kita bisa memasukkan kode program (\f{listing}) ke dalam tugas akhir kita seperti kode Java berikut:
\lstinputlisting[language=Java, caption=Kode sampel Java, label=code:java]{codes/3-sample.java}

\f{Syntax highlighting} kini sudah bisa dilakukan secara otomatis oleh \f{library} yang ada di \latex.
Sudah tidak perlu lagi membuat skrip manual untuk menambahkan \f{syntax highlighting} sendiri.
Cukup definisikan bahasa pemrograman yang digunakan, pada parameter \code{language=} di perintah \code{\bslash{}lstinputlisting}.

Berikut ini adalah daftar bahasa pemrograman yang didukung \f{library} \code{listings}: ABAP, ACSL, Ada, Algol, Ant, Assembler, Awk, bash, Basic, C\#, C++, C, Caml, Clean, Cobol, Comal, csh, Delphi, Eiffel, Elan, erlang, Euphoria, Fortran, GCL, Gnuplot, Haskell, HTML, IDL, inform, Java, JVMIS, ksh, Lisp, Logo, Lua, make, Mathematica, Matlab, Mercury, MetaPost, Miranda, Mizar, ML, Modelica, Modula-2, MuPAD, NASTRAN, Oberon-2, Objective C, OCL, Octave, Oz, Pascal, Perl, PHP, PL/I, Plasm, POV, Prolog, Promela, Python, R, Reduce, Rexx, RSL, Ruby, S, SAS, Scilab, sh, SHELXL, Simula, SQL, tcl, TeX, VBScript, Verilog, VHDL, VRML, XML, XSLT.\cite{latex:source_code_listings}

Satu contoh lagi, sebuah kode bahasa pemrograman Python:
\lstinputlisting[language=Python, caption=Kode sampel Python, label=code:python]{codes/3-sample.py}

Anda juga bisa menambahkan \f{caption} untuk memberikan ringkasan tentang kode tersebut.
Namun, jangan lupa untuk menjelaskan kode melalui paragraf, terutama pada bagian-bagian yang perlu penjelasan lebih.
Penting bagi pembaca untuk memahami mengapa kode tersebut disertakan dalam laporan tugas akhir Anda.

%-----------------------------------------------------------------------------%
\section{Mengubah Tampilan Teks}
%-----------------------------------------------------------------------------%
\begin{itemize}
	\item \code{\bslash{}f} \\
	Merupakan alias untuk perintah \code{\bslash{}textit}, contoh 
	\f{contoh hasil tulisan}.
	\item \code{\bslash{}bi} \\
	\bi{Contoh hasil tulisan}.
	\item \code{\bslash{}bo} \\
	\bo{Contoh hasil tulisan}.
	\item \code{\bslash{}m} \\
	\m{Contoh hasil tulisan}.
	\item \code{\bslash{}mc} \\
	\mc{Contoh hasil tulisan}.
	\item \code{\bslash{}code} \\
	\code{Contoh hasil tulisan}.
\end{itemize}


%-----------------------------------------------------------------------------%
\section{Memberikan Catatan}
%-----------------------------------------------------------------------------%
Ada dua perintah untuk memberikan catatan penulisan dalam dokumen yang Anda kerjakan, yaitu: 
\begin{itemize}
	\item \code{\bslash{}todo} \\
	Contoh: \\ \todo{Contoh bentuk todo.}
	\item \code{\bslash{}todoCite} \\ 
	Contoh: \todoCite
\end{itemize}


%-----------------------------------------------------------------------------%
\section{Menambah Isi Daftar Isi}
%-----------------------------------------------------------------------------%
Terkadang ada kebutuhan untuk memasukan kata-kata tertentu kedalam Daftar Isi.
Perintah \code{\bslash{}addChapter} dapat digunakan untuk judul bab dalam Daftar isi.
Contohnya dapat dilihat pada berkas thesis.tex.


%-----------------------------------------------------------------------------%
\section{Memasukan PDF}
%-----------------------------------------------------------------------------%
Untuk memasukan PDF dapat menggunakan perintah \code{\bslash{}inpdf} yang menerima satu buah argumen.
Argumen ini berisi nama berkas yang akan digabungkan dalam laporan.
PDF yang dimasukan degnan cara ini akan memiliki header dan footer seperti pada halaman lainnya. 

\inpdf{pdfs/include}

Cara lain untuk memasukan PDF adalah dengan menggunakan perintah \code{\bslash{}putpdf} dengan satu argumen yang berisi nama berkas pdf.
Berbeda dengan perintah sebelumnya, PDF yang dimasukan dengan cara ini tidak akan memiliki footer atau header seperti pada halaman lainnya. 

\putpdf{pdfs/include}


%-----------------------------------------------------------------------------%
\section{Membuat Perintah Baru}
%-----------------------------------------------------------------------------%
Ada dua perintah yang dapat digunakan untuk membuat perintah baru, yaitu: 
\begin{itemize}
	\item \code{\bslash{}Var} \\
	Digunakan untuk membuat perintah baru, namun setiap kata yang diberikan akan diproses dahulu menjadi huruf kapital. 
	Contoh jika perintahnya adalah \code{\bslash{}Var\{adalah\}} makan ketika perintah \code{\bslash{}Var} dipanggil, yang akan muncul adalah ADALAH. 
	\item \code{\bslash{}var} \\
	Digunakan untuk membuat perintah atau baru. 
\end{itemize}
Membuat perintah baru sebaiknya dilakukan pada berkas \code{uithesis.sty}.
Berkas \code{uithesis.sty} adalah berkas khusus pengatur \f{styling} untuk tugas akhir ini.
Berkas itu berisikan semua konfigurasi yang dibutuhkan untuk membuat dokumen \latex~ini menjadi sesuai dengan Peraturan Rektor, termasuk perintah-perintah baru.

Jika perubahan ini dirasa penting untuk disertakan dalam template, silakan lakukan \f{fork} \url{repositori Git template ini}.