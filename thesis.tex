%% Template for ENG 401 reports
%% by Robin Turner
%% Adapted from the IEEE peer review template

\documentclass[journal,12pt,onecolumn,a4paper]{IEEEtran}
\usepackage{cite} % Tidies up citation numbers.
\usepackage{url} % Provides better formatting of URLs.
\usepackage[utf8]{inputenc} % Allows Turkish characters.
\usepackage{booktabs} % Allows the use of \toprule, \midrule and \bottomrule in tables for horizontal lines
\usepackage{graphicx}
\usepackage{amsmath}
\usepackage{float}
\usepackage{multicol}
\usepackage{listings}
\usepackage[numbered,framed]{matlab-prettifier}
\usepackage[export]{adjustbox}

\graphicspath{ {./images/} }

\hyphenation{op-tical net-works semi-conduc-tor} % Corrects some bad hyphenation 
\lstset{
  basicstyle=\ttfamily,
  columns=fullflexible,
  breaklines=true,
  postbreak=\mbox{\textcolor{red}{$\hookrightarrow$}\space},
}


\begin{document}
\begin{titlepage}
	% paper title
	% can use linebreaks \\ within to get better formatting as desired
	\title{Interpolasi Numerik: Path Calculator}


	% author names and affiliations

	\author{Adrian Ardizza\\
		Alya Azhar Agharid\\
		Muhammad Athallah\\
		Stefanus Ndaru Wedhatama
	}

	% make the title area
	\maketitle
	\begin{abstract}
		The abstract does not only mention the paper but is the original paper shrunken to approximately 200 words. It states the purpose, reports the information obtained, and gives conclusions, and recommendations. In short, it summarizes the main points of the study adequately and accurately. It provides information from every major section in the body of the report in a dense and compact way. Past tense and active voice is appropriate when describing what was done. If there is any, it includes key statistical detail.

		Depending on the format you use, the abstract may come on the title page or at the beginning of the main report.

	\end{abstract}
	\tableofcontents
	\listoffigures
	\listoftables
\end{titlepage}

\IEEEpeerreviewmaketitle

\section{Pendahuluan}
Interpolasi merupakan teknik yang digunakan untuk membangun suatu fungsi yang melewati sebuah himpunan titik-titik diskret yang diketahui. Fungsi interpolasi memiliki aplikasi yang luas pada Ilmu Komputer karena dapat digunakan untuk menggambarkan berbagai kurva kompleks dengan \emph{computational cost} yang relatif murah apabila dibandingkan dengan metode brute-force (mencari seluruh titik yang memenuhi suatu kurva secara \emph{exhaustive}). Interpolasi beberapa titik diskret dapat dicapai dengan berbagai metode interpolasi seperti interpolasi konstan, interpolasi linear, interpolasi polinomial, dan interpolasi \emph{spline} yang menjadi topik dari paper ini.

Secara matematis, sebuah fungsi interpolasi dapat dinyatakan sebagai suatu fungsi \(P(x)\) dimana \(x\) adalah sebuah titik yang ingin di-interpolasi. Hasil dari fungsi tersebut adalah sebuah value yang merupakan aproksimasi dari nilai fungsi \(f(x)\) yang melewati seluruh titik-titik diskret yang nilainya sudah diketahui.

Pada paper ini, kami ingin mengeksplorasi pemanfaatan sebuah fungsi \emph{cubic spline} - atau lebih tepatnya fungsi \emph{cubic spline} dengan kondisi \emph{natural} - untuk melakukan \emph{smoothing} pada perubahan posisi suatu titik di interval waktu \([0, n]\). Cubic spline dapat didefinisikan sebagai suatu \emph{piecewise polyomial function} yang menginterpolasi sebuah titik
\section{Metode Integrasi Romberg}
Hello
\subsection{Pengenalan Algoritma Romberg}
\emph{Romberg Integration} merupakan metode integrasi yang menggunakan prinsip \emph{trapezoidal rule} dan meningkatkan akurasi dengan melakukan ekstrapolasi melalui \emph{Richardson's Extrapolation}.

\emph{Romberg} bekerja dengan menggunakan \emph{Trapezoidal Rule} sebagai basis dari perhitungan dan melakukan iterasi dengan melakukan pembagian interval untuk meningkatkan akurasi. Basis ini diiterasi per-baris pada suatu tabel Romberg yang memiliki ukuran \(R(n,n)\) melalui rumus

\begin{equation*}
	\begin{split}
		R(1,1) & = (b-a)\frac{f(a)+f(b)}{2} \\
		R(j,1) & = \frac{1}{2}R_{j-1,1}+h_j\sum_{i=1}^{2^{j-2}}f(a+(2i-1)h_j)
	\end{split}
\end{equation*}

Jika dilihat, \(R(1,1)\) perlu diinisiasi terlebih dahulu karena untuk iterasi berikutnya, akan menggunakan hasil dari iterasi sebelumnya karena bersifat \emph{evolutionary} dengan arti meningkatkan akurasi dari iterasi sebelumnya dengan membagi interval \(h\) menjadi lebih kecil. Nilai \(h\) ini diperkecil setiap iterasi melalui rumus

\begin{equation*}
	\begin{split}
		h_j & = \frac{b-a}{2^{j-1}}
	\end{split}
\end{equation*}

Dari sini, kita bisa membentuk tabel Romberg \(R\) yang memiliki iterasi \emph{trapezoidal rule} dengan iterasi setiap baris, di mana semakin banyak iterasi (dan baris semakin ke bawah), maka akurasi akan semakin meningkat karena memiliki jarak interval \(h\) yang semakin kecil.

Untuk setiap baris, kita bisa menemukan akurasi yang lebih tinggi dengan mengekstrapolasi menggunakan \emph{Richardson's Extrapolation} yang mengekstrapolasi hasil dari \emph{Trapezoidal Rule} yang diperoleh. Rumus \emph{Richardson's Extrapolation} yang sudah diadaptasi untuk algoritma Romberg ini dapat dilihat sebagai berikut

\begin{equation*}
	\begin{split}
		R_{j,k} & = \frac{4^{k-1}R_{j,k-1}-R_{j-1,k-1}}{4^{k-1}-1}
	\end{split}
\end{equation*}

Dengan demikian, kita bisa mengerti bahwa algoritma Romberg melakukan integrasi dengan membangun tabel Romberg \(R\) dengan ukuran \(R(n,n)\) dan mengisi tabel tersebut sehingga membentuk \emph{lower triangular matrix} di mana setiap baris merupakan iterasi \emph{trapezoidal rule} dan setiap kolom merupakan peningkatan akurasi dari \emph{trapezoidal rule} baris tersebut menggunakan \emph{richardson's extrapolation} dengan meningkatkan \emph{order of accuracy} melalui hasil yang diperoleh pada iterasi-iterasi sebelumnya (yang akan dibahas pada \emph{subsection} toleransi eror di bawah). Dengan demikian, sebuah tabel Romberg yang dihasilkan dapat divisualisasikan sebagai berikut

\begin{center}
	\begin{tabular}{ c | c c c c c }

		j/k        & \(R(j,0)\) & \(R(j,1)\) & \(R(j,2)\) & \(R(j,3)\) & \(\cdots\) \\
		\hline
		0          & \(R(0,0)\) & -          & -          & -          & -          \\
		1          & \(R(1,0)\) & \(R(1,1)\) & -          & -          & -          \\
		2          & \(R(2,0)\) & \(R(2,1)\) & \(R(2,2)\) & -          & -          \\
		3          & \(R(3,0)\) & \(R(3,1)\) & \(R(3,2)\) & \(R(3,3)\) & -          \\
		\(\vdots\) & \(\vdots\) & \(\vdots\) & \(\vdots\) & \(\vdots\) & \(\ddots\) \\
	\end{tabular}
\end{center}

Sesuai dengan prinsip yang dijelaskan di atas, akurasi ditingkatkan dengan ekstrapolasi setiap kolomnya dengan semakin meningkat kolom (semakin ke kanan), maka hasil akan semakin akurat. Dengan demikian, nilai paling akurat untuk setiap iterasi berada pada diagonal matriks tersebut. Maka dari itu, untuk memperoleh hasil integrasi, diambil elemen diagonal pada baris terakhir, yaitu \(R(j,j)\). Dengan demikian, dapat disimpulkan hasil integrasi melalui Romberg sebagai
\begin{equation*}
	\begin{split}
		\int_{a}^{b} f(x) \,dx & = R(j,j)
	\end{split}
\end{equation*}

\subsection{Implementasi Algoritma Romberg}
Untuk tugas kelompok kedua ini, diberikan fungsi distribusi normal tabel Z sebagai berikut
\begin{equation*}
	\begin{split}
		P(Z <= z ) & = \phi(z) = \int_{-\infty}^{z} \frac{1}{\sqrt{2\pi}}e ^{\frac{-z^2}{2}} \,dz
	\end{split}
\end{equation*}



\section{Kesimpulan}
Kesimpulan disini.

% % Example of a table from http://www.latextemplates.com/template/professional-table

\begin{thebibliography}{1}
	% Here are a few examples of different citations 
	% Book
	\bibitem{kopka_1999} % Note the label in the curly brackets. Use the cite the source; e.g., \cite{kopka_latex}
	H.~Kopka and P.~W. Daly, \emph{A Guide to \LaTeX}, 3rd~ed.\hskip 1em plus
	0.5em minus 0.4em\relax Harlow, England: Addison-Wesley, 1999.
	\bibitem{horowitz_2005}D.~Horowitz, \emph{End of Time}. New York, NY, USA: Encounter Books, 2005. [E-book] Available: ebrary, \url{http://site.ebrary.com/lib/sait/Doc?id=10080005}. Accessed on: Oct. 8, 2008.
	% Article from database
	\bibitem{castlevecchi_2008}D.~Castelvecchi, ``Nanoparticles Conspire with Free Radicals'' \emph{Science News}, vol.174, no. 6, p. 9, September 13, 2008. [Full Text]. Available: Proquest, \url{http://proquest.umi.com/pqdweb?index=52&did=1557231641&SrchMode=1&sid=3&Fmt=3&VInst=PROD&VType=PQD&RQT=309&VName=PQD&TS=1229451226&clientId=533}. Accessed on: Aug.~3, 2014.
	% Conference Paper from the Internet
	\bibitem{lach_2010}J.~Lach, ``SBFS: Steganography based file system,'' in \emph{Proceedings of the 2008 1st International Conference on Information Technology, IT 2008, 19-21 May 2008, Gdansk, Poland.} Available: IEEE Xplore, \url{http://www.ieee.org}. [Accessed: 10 Sept. 2010].
	% Web page, no author
	\bibitem{a_laymans_explanation}``A `layman's' explanation of Ultra Narrow Band technology,'' Oct.~3, 2003. [Online]. Available: \url{http://www.vmsk.org/Layman.pdf}. [Accessed: Dec.~3, 2003].
\end{thebibliography}

% This is a hand-made bibliography. If you want to use a BibTeX file, you're on your own ;-)














\end{document}