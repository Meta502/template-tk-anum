%-----------------------------------------------------------------------------%
\chapter{\babLima}
\label{bab:5}
%-----------------------------------------------------------------------------%
Apa itu Bab 5?

\todo{Tuliskan paragraf pengantar bab lima di sini. Bab lima pada tugas akhir S1 umumnya merupakan pembahasan analisis dari penelitian. Namun, sekali lagi, sesuaikan dengan kebutuhan Anda. Tesis atau disertasi tentunya berbeda dengan skripsi.}


%-----------------------------------------------------------------------------%
\section{Metode Analisis}
\label{sec:method}
%-----------------------------------------------------------------------------%
Metode analisis yang dilakukan dalam penelitian ini adalah sebagai berikut:
\begin{itemize}
	\item Tahap 1.
	\item Tahap 2.
\end{itemize}


%-----------------------------------------------------------------------------%
\section{Hasil Analisis}
\label{sec:analisis}
%-----------------------------------------------------------------------------%
Berdasarkan analisis yang dilakukan pada \f{hardware} dengan spesifikasi ...., berikut ini adalah performa pembuatan produk yang dilakukan secara bersamaan:
\begin{table}
	\centering
	\begin{tabular}{|l|c|c|c|c|}
		\hline
		Jumlah   & \f{CPU Time} (s) & CPU \% / \f{core} & Waktu (s) & Memori (MB) \\ \hline
		1 produk & 101,21           & 35                & 37,86     & 917,35      \\ \hline
		5 produk & 120,728          & 17,18             & 87,46     & 915,28      \\ \hline
	\end{tabular}
	\caption{Contoh Tabel: Analisis performa pembuatan produk ketika dijalankan bersamaan}
	\label{table:sample}
\end{table}

\todo{Tulis penjelasan terkait \tab~\ref{table:sample} di sini. Jika Anda hanya menunjukkan data, pembaca tidak akan tahu apakah data tersebut berharga.}
